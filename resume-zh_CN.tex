% !TEX TS-program = xelatex
% !TEX encoding = UTF-8 Unicode
% !Mode:: "TeX:UTF-8"

\documentclass{resume}
\usepackage{zh_CN-Adobefonts_external} % Simplified Chinese Support using external fonts (./fonts/zh_CN-Adobe/)
%\usepackage{zh_CN-Adobefonts_internal} % Simplified Chinese Support using system fonts
\usepackage{linespacing_fix} % disable extra space before next section
\usepackage{cite}
\usepackage{graphicx}

\begin{document}
\pagenumbering{gobble} % suppress displaying page number

\name{雍昭宇}

% {E-mail}{mobilephone}{homepage}
% be careful of _ in emaill address
\contactInfo{(+86) 188-1926-8108}{yongzhaoyu@qq.com}{后端开发工程师}{\textit{https://github.com/yzhaoyu}}
% {E-mail}{mobilephone}
% keep the last empty braces!
%\contactInfo{xxx@yuanbin.me}{(+86) 131-221-87xxx}{}

%\section{个人总结}

% \section{\faGraduationCap\ 教育背景}
\section{教育背景}
\datedsubsection{\textbf{重庆大学 (硕士)},电子与通信工程 (推免)}{2018.09 - 2021.06}
\begin{itemize}
  \item 重庆大学A等学业奖学金 (2018和2019年度)
  \item “兆易创新杯”第十四届中国研究生电子设计竞赛西南分赛区团队二等奖
  \item 重庆大学第六届研究生智慧城市技术与创意设计大赛一等奖
\end{itemize}
\datedsubsection{\textbf{华南农业大学 (本科)},通信工程}{2014.09 - 2018.06}
\begin{itemize}
  \item 国家励志奖学金(Top 5\%,2014—2016)
  \item 华南农业大学2014—2015年度校级二等奖学金,获得“三好学生”荣誉称号
  \item 华南农业大学2015—2017年度校级三等奖学金
\end{itemize}

% \section{\faCogs\ IT 技能}
\section{专业技能}
% increase linespacing [parsep=0.5ex]
\begin{itemize}[parsep=0.2ex]
  \item 熟悉Golang,了解Golang内部运行机制,熟悉goroutine、channel和锁等机制。
  \item 熟悉Python,熟悉Django web开发框架、MVC模型和RESTful API的开发。
  \item 熟悉常用数据结构与算法,能够实现哈希表、链表、队列、栈、二叉树等数据结构。
  \item 熟悉操作系统和TCP/IP协议。
  %\item 了解微服务与RPC,有一定的使用经验。
  \item 了解Linux网络编程、Linux系统及常用命令和Docker容器。
  \item 熟悉Redis,了解MySQL关系型数据库,了解NSQ和Kafka消息队列。
\end{itemize}

% \end{itemize}

\section{实习经历}
\datedsubsection{\textbf{腾讯科技(深圳)有限公司 | Tencent},后端开发工程师}{2020.04-2020.09}
\begin{onehalfspacing}
本人任职于腾讯增值研发部门,岗位是后端开发,负责QQ黄钻业务、QQ会员等与增值营收相关的会员包月服务后台架构设计和开发维护,主要参与项目:
\begin{itemize}
  \item \textbf{天璇活动平台、天璇活动平台监控告警系统。}
  \item \role{技术选型  Golang,  tRPC,  MySQL,  Redis,  Protobuf}{}
  \item 主要工作:1.使用Golang构建天璇活动平台后端,通过tRPC、Protobuf与前端、其他后端进行交互。2.设计数据库表,使用MySQL存储活动名称、活动ID、活动归属人等数据,将活动配置、每日消耗等数据存储到Redis中。3.为保证活动平台可靠运行,开发了配套的监控告警系统,分为两个部分:库存监控告警和线上错误实时告警,当库存不足或线上有错误时通过企业微信提供的API实现自动建群通知活动归属人。
\end{itemize}
\end{onehalfspacing}

\datedsubsection{\textbf{重庆博尼施科技有限公司 | Burnish},后端开发工程师}{2019.09-2019.12}
\begin{onehalfspacing}
公司属于车联网监控服务提供商,为新能源和传统能源车提供满足国家标准即国五、国六标准和地方标准的数据上报、数据转发和车辆的监控服务,主要参与项目:
\begin{itemize}
  \item \textbf{东风新能源汽车监控系统和数据网关。}
  \item \role{技术选型  Golang,  Python,  Django,  Django REST framework,  MySQL,  Redis,  Kafka}{}
  \item 主要工作:1.使用Python、Django和Django REST framework采用前后端分离的方式,构建一个高可用、高并发、高负载的系能源汽车监控系统。2.使用Golang编写了一套数据网关,用来接收国六标准数据、新能源汽车数据,负责数据的接收、数据包分割和验证、向Kafka消息队列集群发送数据。3.使用Docker部署新能源汽车监控系统与数据网关,Kuebrnetes进行容器的编排与管理。
\end{itemize}
\end{onehalfspacing}

\section{个人总结}
\begin{enumerate}
  \item 本人在校成绩优秀、乐观向上,工作负责、自我驱动力强、热爱尝试新事物,热爱互联网行业。
  \item 熟悉Golang和Python,最近一年一直在做WEB方面的研发工作。
\end{enumerate}

% \begin{onehalfspacing}
% \end{onehalfspacing}

%% Reference
%\newpage
%\bibliographystyle{IEEETran}
%\bibliography{mycite}
\end{document}
